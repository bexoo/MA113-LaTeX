\documentclass[12pt]{article}

% Include math
\usepackage{amsmath,amsthm,amssymb}
% Include links
\usepackage{hyperref}

%%%%%%%%%%%%%  THEOREMS  %%%%%%%%%%%%%%%%%
\theoremstyle{plain} % other options: definition, remark
\newtheorem{theorem}{Theorem}
\newtheorem{lemma}[theorem]{Lemma}
% By including [theorem], the lemma follows the numbering of theorem
% e.g. Thm 1, Lemma 2, Thm 3, Thm 4, \dots
\theoremstyle{definition}
\newtheorem*{definition}{Definition} % the star prevents numbering

% Remarks
\theoremstyle{remark}
\newtheorem{remark}{Remark}




%%%%%%%%%%%%%%  PAGE SETUP %%%%%%%%%%%%%%%%%
% LaTeX has big default margins
% The following sets them to 1in
\usepackage[margin=1.5in]{geometry}

% The following sets up some headers
\usepackage{fancyhdr}
\pagestyle{fancy}
\lhead{MA113 Notes} % Left Header
\rhead{\thepage} % Right Header
\cfoot{} % Center Foot (empty)






%%%%%%%%%%%%% SHORTCUTS %%%%%%%%%%%%%%%%%%%%
\newcommand{\half}{\frac{1}{2}}
\newcommand{\cbrt}[1]{\sqrt[3]{#1}}

% Document content begins here
\begin{document}

% Set up a title
\title{MA113 Notes}
\author{Brady Exoo}
\date{\today}
\maketitle

\begin{abstract}
Notes for the MA 113 (Mutlivariable Calculus) taught by Dr. Holder.
\end{abstract}

% This line makes a ToC
\tableofcontents

% This line starts a new page
\eject

\section{Coordinate Systems and Polar Functions}
\subsection{Rectangular Coordinates}
\begin{itemize}
    \item $(x,y)$
    \item Every point has only one set of coordinates
    \item Unique representation!
\end{itemize}
\subsection{Polar Coordinates}
\begin{itemize}
    \item $(r,\theta)$
    \begin{itemize}
        \item $r$ is the distance from the origin
        \item $\theta$ is the angle the radius faces
    \end{itemize}
    \item Lacks unique representation
    \begin{itemize}
        \item Can add $2\pi$ to any angle and get the same point
        \item Can make radius negative and add $\frac{\pi}{2}$ to the angle and get the same point
    \end{itemize}
\end{itemize}
These coordinates are helpful for setting up integrals later.
\subsection{Transformations}
\definition{$\tan{\theta}$ is the distance from the intersection of the extended radius and the vertical tangent to the tangent point.}
\begin{itemize}
    \item $x^2+y^2=r^2$
    \item $\tan{\theta} = \frac{y}{x}$
    \begin{itemize}
        \item $x=0 \implies \theta \in \{\frac{\pi}{2}, \frac{3\pi}{2}\}$
    \end{itemize}
    \eject
    \item $\theta = \tan^{-1}{\frac{y}{x}}$
    \begin{itemize}
        \item $\tan{\theta}$ is the distance from the intersection of the extended radius and the vertical tangent to the tangent point
        \item This is problematic! Range of arctan is $(\frac{-\pi}{2},\frac{\pi}{2})$, so we do not get all $360$ degrees.
        \item $ \theta = \begin{cases}
            \tan^{-1}{\frac{y}{x}} & x\leq 0 \\
            \frac{\pi}{2} & x=0,y>0 \\
            \frac{3\pi}{2} & x=0,y<0 \\
            \tan^{-1}{\frac{y}{x}} + \pi & x < 0
            \end{cases}
        $
    \end{itemize}
    \item $r\cos{\theta} = x$
    \item $r\sin{\theta} = y$
\end{itemize}
\subsection{Polar Graphs}
\begin{itemize}
    \item Graphs are of the form $r(\theta)$, $r$ is a function of $\theta$.
    \item $r(\theta) = \cos{\theta}-1$ is a cardioid.
    \item $r(\theta)=\cos{(2\theta)}$ is a $4-$flower.
    \begin{itemize}
        \item $\cos{(\text{n}\theta)}$ has $2n$ petals if $n$ is even, $n$ petals if $n$ is odd.
    \end{itemize}
\end{itemize}
\subsection{Differentiating Polar Functions}
\begin{itemize}
    \item We come to a problem, we wish to find $\frac{dy}{dx}$ but our functions are in terms of $r$ and $\theta$!
    \item $\frac{dy}{dx}$ = $\frac{dy/d\theta}{dx/d\theta}$ by the Chain Rule.
    \begin{itemize}
        \item Not because we can cancel out the $d\theta$ terms!
    \end{itemize}
\end{itemize}
\subsubsection{Chain Rule}
The default way the Chain Rule is portrayed is
\[ \frac{d}{dx} f(g(x)) = f^\prime(g(x)) \cdot g^\prime(x) \text{.} \]
Other notation is
\[ \frac{df}{dx} = \frac{df}{dg} \cdot \frac{dg}{dx} \text{.} \]
The Chain Rule allows us to change variables which we do not wish to differentiate by.
\subsubsection{Polar Derivatives}
Now we can substitute $y$ and $x$ with our earlier transformations to get
\[ \frac{dy}{dx} = \frac{\frac{d}{d\theta}r\sin{\theta}}{\frac{d}{d\theta}r\cos{\theta}} \text{.} \]

\subsection{Arc Lengths}
In Calculus II, we learned the arclength of a function from $a$ to $b$ as
\[ \int_a^b{\sqrt{1+(f^\prime(x))^2}dx} \]
This is somewhat related to
\[ \frac{d}{dx} \langle x,f(x) \rangle = \langle 1,f^\prime(x) \rangle \]
\[ || \langle1, f^\prime(x) \rangle || = \sqrt{1+(f^\prime(x))^2} \text{.} \]
We are essentially integrating the magnitude of the derivative of the vector, which gives us the arclength (think about this visually with the Pythagorean Theorem).















\end{document}
